\documentclass[10pt, letterpaper, titlepage]{article} % Set font here.
% Use 'article' for simple documents; use 'report' for larger documents with chapters;
% use 'book' for even larger documents with parts.
\usepackage[utf8]{inputenc}
\usepackage{geometry}
\usepackage{color,graphicx,overpic} 
\usepackage{fancyhdr} % header/footer stuff
\usepackage{amsmath,amsthm,amsfonts,amssymb}
\usepackage{mathtools} % more math stuff
\usepackage{siunitx} % for SI units, ex. $3.5 ~ \si{kg.s^{-2}}$
\usepackage{hyperref} % for hyperlinks
%\usepackage{apple_emoji}
\usepackage{multicol}
\usepackage{array}
\usepackage{float}
\usepackage{blindtext}
\usepackage{longtable}
\usepackage{scrextend}
\usepackage[font=small,labelfont=bf]{caption}
\usepackage[framemethod=tikz]{mdframed}
\usepackage{calc}
\usepackage{titlesec}
\usepackage{listings}
\usepackage[normalem]{ulem}

\definecolor{mycolor}{rgb}{0, 0, 0}

\pdfinfo{
  /Title (bkongLab4.pdf)
  /Creator (Benjamin Kong)
  /Producer (pdfTeX 1.40.0)
  /Author (Benjamin Kong)
  /Subject (Resume)
  /Keywords (pdflatex, latex,pdftex,tex)}
  

\geometry{top=2.7cm,left=1.8cm,right=1.8cm,bottom=2.7cm}
\setlength{\headheight}{17pt}
\renewcommand{\baselinestretch}{1.5} 
\setlength{\parskip}{0.3cm}
\setlength{\parindent}{0.6cm}
\titlespacing\section{0pt}{12pt plus 4pt minus 2pt}{0pt plus 2pt minus 2pt}

\newcommand{\barrows}{\textcolor{blue}{\Longrightarrow}\quad}
\newcommand{\barrow}{\quad\textcolor{blue}{\Longrightarrow}\quad}  
\newcommand{\sumi}[1][1]{ \sum_{n={#1}}^{\infty} }
\newcommand{\limi}[1][n]{ \lim_{{#1}\to\infty} }

\title{\textbf{\Huge{
\begin{center}
Introduction to\\ Assembly Language\\
\end{center}
}}}
\author{Benjamin Kong || 1573684\\Lora Ma ||||| 1570935\\ \\ECE 212 Lab Section H11}

\pagestyle{fancy}
\fancyhf{}
\rhead{Benjamin Kong \& Lora Ma}
\lhead{\textit{Introduction to Assembly Language}}
\rfoot{Page \thepage}

\begin{document} 
\pagenumbering{gobble} 
\maketitle 
\thispagestyle{empty}
\tableofcontents 
\newpage
\pagenumbering{arabic}

\begin{multicols*}{2}

\section{Introduction}

Assembly language is a low-level programming language that is converted to machine code using an assembler. 
Assembly language uses \textit{mnemonics} to represent low-level machine instructions, and this makes it much more readable than machine code.
It is important to note that assembly language is specific to a particular computer architecture and hence may or may not work on different systems.

The purpose of this lab was to become more familiar with assembly language using the NetBurner ColdFire microcontroller board. 
In order to gain experience with assembly language programming, two different programs were created using assembly. 
In the first part of the lab (part A), we created a program that converted an Ascii character to its hexadecimal equivalent. 
For example, `5' is converted to `5' and `B' is converted to `11.'
However, if a character without a hexadecimal equivalent is entered, such as `J,' then an error code results.

For the second part of the lab (part B), we created a program that converts an Ascii letter into its uppercase or lowercase equivalent.
For example, `a' is converted to `A' while `E' is converted to `e.'
If an invalid Ascii character is input (such as an ampersand `\&,' for example), an error code results.

\section{Design}

\subsection{Part A}

\subsection{Part B}




\end{multicols*}

\end{document}
